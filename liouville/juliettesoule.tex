

\subsection{By Juliette Soule}
{\bf Implications of Fundamental Theorem of Statistical Mechanics for Calculating Macroscopic Properties from a Microscopic System}

Liouville's Theorem in application to Hamiltonian systems states that the density of states in an ensemble of many identical states with different initial conditions is constant along every trajectory in phase space. For a system described by the Hamiltonian $H = H(q_{1},...,q_{n},p_{1},...p_{n})$ where $q_{i}$ and $p_{i}$ for $i=1...n$ are the canonical positions and momenta respectively, the phase space distribution function $\rho(p,q)$ determines the probability $\rho(p,q)d^{n}qd^{n}p$ that the system will be found in the (infinitesimal) phase space volume $d^{n}qd^{n}p$. The formal statement of Liouville's theorem is then.
\begin{equation}
\frac{d\rho}{dt} = \frac{\partial\rho}{\partial t} + \Sigma_{i}^{n}(\frac{\partial\rho}{\partial q_{i}}\dot{q_{i}}+\frac{\partial\rho}{\partial p_{i}}\dot{p_{i}}) = 0
\end{equation}
Liouville's Theorem can equivalently be formulated as the statement that the flow of a Hamiltonian system preserves phase space volume
In Statistical Mechanics, any particular macrostate is represented as a statistical ensemble associated with a large number of distinct possible microstates, each of which is a point in phase space with their probability distribution given by the phase space distribution function. 
Since in an equilibrium system $\frac{\partial\rho}{\partial t} = 0$, and by Liouville's Theorem $\frac{d\rho}{dt} = 0$, we therefore can conclude that $\{\rho, H\}=0$ in an equilibrium system. This implies that the phase space distribution function must be a pure function of the Hamiltonian, ie $\rho(p,q) = F(H(p,q))$. Integrating over all of phase space we obtain the \textit{partition function} $\mathcal{F} = \int d^{n}pd^{n}qF(H(p,q))$. The partition function is the quantity from which all thermodynamic quantities are derived, indicating the essential role of Liouville's Theorem in the calculation of macroscopic properties from the microscopic system. Furthermore, Liouville's Theorem is the reason why it is possible to divide up continuous phase space into discrete volumes/microstates, and thus do Statistical Mechanics. Suppose that this were \textit{not} possible, and consider a density of states, associated with a region of volume V in phase space corresponding to a particular number of microstates. If Liouville's Theorem did not hold, we can imagine that as the region evolved its volume would increase. Consider the total probability P for the system to be inside that region. The probability density in the region is initially P/V. As it grows, its probability density decreases; meanwhile other regions of phase space shrink, so their probability density increases. This is in violation of one of the fundamental assumptions of Statistical Mechanics, the postulate of equal \textit{a priori} probabilities: that for a system in equilibrium, every microstate consistent with the macrostate has an equal probability. That is, all regions of phase space consistent with the macrostate have an equal probability density. If Liouville's Theorem did not hold, then as described it would be possible for probability density to increase in some regions and decrease in others, and therefore this assumption would be grossly violated, thus undermining the entirety of Statistical Mechanics.

