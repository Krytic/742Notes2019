
\subsection{By Elias Paknejad}

In statistical physics, the microscopic state of a system is described by an ensemble of states that have different initial conditions within a defined precision. Every member of the ensemble corresponds to a point in the phase space. Every point in the phase space evolves in time according to the Hamiltonian.

As we have seen in the lecture, the central problem of thermodynamics is the find the final thermodynamic state of equilibrium given some conditions. In describing a thermodynamic system, entropy plays an important role, because it connects the  macroscopic world to the microscopic properties:

\begin{equation}
S = k \ln{\Omega},
\end{equation}

where $k$ is the Boltzmann's constant and $\Omega$ the phase space volume.

Liouvilles's theorem states that as the system evolves with time according to the Hamiltonian, the volume $\Omega$ in phase space that is shaped by the accessible microstates does not change:

\begin{equation}
\frac{d\Omega}{dt} = 0,
\end{equation}


The interesting point is that the microstates of the system do change constantly because of the motion of the particles according to the equations of motions. But the macroscopic properties like $S$ will remain constant if nothing special happens. So the Liouvilles's theorem shows us that we don't have to know the motion of every particle to be able to determine the macroscopic state of the system. 

Liouvilles's theorem  also implies that the probability of finding a the system in a volume element in the phase space must also remain constant. So by knowing the probability density $\rho_0$ at $t=0$, we can calculate the probability of finding the system in a specific volume in phase space at any time as the system evolves.
