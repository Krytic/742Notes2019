\subsection{By Ryan Stickland} 
{\bf Liouville's Theorem and its implications for statistical mechanics.}

Liouville's Theorem states, in general terms, that a Hamiltonian system preserves its volume in phase space in time. In terms relevant to statistical mechanics it says that, if you have a macroscopic, Hamiltonian system of many particles, observing the path of any one particle is approximately equivalent the path of all particles as $t\to\infty$. For a Hamiltonian system with canonical co-ordinates $p$ and $q$ the equation that represents this, Liouville’s equation, is as follows:

$$\frac{\partial\rho}{\partial t}=-\{\rho,\mathcal{H}\}=0$$

Where $\rho(p,q,t)$ is the phase space distribution function and $\mathcal{H}$ is the Hamiltonian for the system. $\rho$ gives the probability that the system will be found in a certain state. This is an immensely important result for statistical mechanics. It is the reason statistical mechanics can describe the behaviour of small microsystems and still be applicable to macroscopic classical systems. It lets us make the assumption that, as the size of the system we are observing grows, the probability density inside remains constant. Assumptions we make for observations in classical mechanics are impossible without this assumption.
