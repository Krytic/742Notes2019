\subsection{Itwinder Singh}
In a phase space each point represents a micro system composed of N particles. Assume a small volume dV associated with it. At any given time each particle has it's specified position and momentum coordinates so the phase space has 6N dimensions
$$\Gamma=\Pi_{i=1}^N (\Vec{q}_i,\vec{p}_i) $$. 
The system evolves in time according to hamiltonian equations:

\begin{equation}
    \frac{d\vec{q}_i}{dt}=\frac{\partial{H}}{\partial\vec{p}_i},  
    \frac{d\vec{p}_i}{dt}=\frac{\partial{H}}{\partial\vec{q}_i}
\end{equation}

The volume dV associated with the micro system in future is unchanged. It's shape may change but overall volume remains constant. It can be thought of as the number of particles in the volume stay fixed or the inflow of particles into the volume matches the outflow. This is the postulate of Liouville's theorem that the phase space volume is comparable to the density of an incompressible fluid. 
This density is constant on surfaces of constant energy H in phase space. The macro state being a representation of all these micro systems of constant density is a basic assumption of Statistical Mechanics.

    

 
\end{document}

